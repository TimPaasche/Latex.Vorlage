
% !TeX root = ./main.tex

%###################################################################################################
%#                                         DOCUMENT CLASS                                          #
%###################################################################################################

\NeedsTeXFormat{LaTeX2e}
\documentclass[11pt, a4paper, twoside, german, final, openany]{memoir}


%###################################################################################################
%#                                            PACKAGES                                             #
%###################################################################################################
\usepackage[utf8]{inputenc}                     % Required for inputting international characters
\usepackage[T1]{fontenc}                        % Output font encoding for international characters
\usepackage{xcolor}                             % for named colors
\usepackage{eso-pic}                            % required to place huge uni logo at the back - on title page
\usepackage{graphicx}                           % for pictures
\usepackage[hidelinks]{hyperref}                % for hyperlinks 
\usepackage[english, ngerman]{babel}            % english and german spelling correction
\usepackage[babel, german=quotes]{csquotes}     % german quotation
\usepackage{geometry}                           % allows to change the layout of a page in a very simple way
\usepackage[most]{tcolorbox}                    % provides an environment for coloured and framed text boxes
\usepackage{titletoc}                           % alternative headings for toc/lof/lot
\usepackage{tikz}                               % allows the creation of graphics
\usepackage{acronym}                            % provides logic to implement a list of acronyms
\usepackage{enumitem}                           % allows listing
\usepackage[style=alphabetic,
    backend=bibtex]{biblatex}                   % backend logic
\usepackage{float}                              % improves the interface for defining floating objects such as figures and tables.
\usepackage{subfig}                             % allows to more figures to be grouped


